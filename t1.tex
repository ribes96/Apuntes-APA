\documentclass[a4paper,10pt]{article}
\usepackage[utf8]{inputenc}
\usepackage[spanish]{babel}
\usepackage{ marvosym }

%opening
\title{Apuntes Tema 1}
\author{Albert Ribes Marzá}

\begin{document}

\maketitle

\begin{abstract}
Los apuntes que vaya tomando en clase
\end{abstract}

\section{}
\subsection{Introducción a ML}
\paragraph{Ejemplo 1}
Se pretende medir la temperatura ($t$) en un punto de una central nuclear, pero la temperatura es tan alta que no se puede medir directamente con ningún sensor. Se intentará deducir la temperatura así:
\begin{itemize}
\item $t$ - temperatura a predecir (variable)
\item $x$ - vector de variables medibles que posiblemente inciden en $t$
\item $z$ - vector de variables \textbf{NO medibles} que posiblemente inciden en $t$
\end{itemize}

La relación completa es $ t = \delta(x,z)$, que es una función.

Pero no conocemos $z$ \MVRightarrow Aun conociendo $x$, el valor de $t$ oscila. La relación entre $t$ y $x$ se hace \textit{estocástica}

$p(x,t)$ será la probabilidad de que con esa $x$ se tenga esa temperatura.

Hay que construir una función $t = y(x)$ donde $t$ sea el valor más plausible. El problema es que no conocemos $p$

La forma de atacar el problema será recolectar datos $\{(x_1,t_1),(x_2,t_2), \dots ,(x_n,t_n)\} = D$

$(x_n,t_n) \sim^{i.i.p.} p$ (variables independientes e identicamente distribuidas)


El objetivo del \textit{Machine Learning} (ML) es obtener $y$ a partir de $D$. En este caso es un problema de \textit{regresión}



\paragraph{Ejemplo 2}
Se tiene una planta de reciclaje, y se quieren clasificar los objetos que van pasando por la cinta. Los datos son:
\begin{itemize}
\item $t$ - tipo de producto
\item $x$ - los atributos de los productos que captamos con una cámara
\item $z$ - los atributos que no captamos con la cámara
\end{itemize}

\end{document}
