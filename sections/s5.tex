\section{Clasificadores lineales}
Una clasificación es una función $y: \mathbb{R}^d \Rightarrow \{1 \dots k\}$
que particiona $\mathbb{R}^d$ en $k$ subespacios.

Definimos una región $R_k$ como $R_k = \{x \in \mathbb{R}^d | y(x) = k \}$. Las
separaciones entre regiones se llaman \textit{fronteras} o ``superficies
de decisión''. Un modelo lineal de clasificación se caracteriza porque las
fronteras que genera son hiperplanos (en $\mathbb{R}^{d - 1}$)

Hay muchos tipos de clasificadores, pero los que comunmente son más
interesantes son los probabilísticos, i.e, clasificadores que dado un
dato $x$ indican la probabilidad de pertenencia a cada una de las regiones.

Puesto que queremos clasificadores lineales, seguiremos usando el método de
multiplicar y sumar la entrada con pesos, pero ahora tenemos la condición de
que el resultado sea una probabilidad, y para ello el primer paso es que esté
acotado en el intervalo $(0,1)$.

Nuestro problema de momento será encontrar una función $g$ de la forma
$x \Rightarrow y(x) = g(w^Tx + w_0)$, donde $g: \mathbb{R} \Rightarrow (0,1)$

A esta función le vamos a exigir que sea continua, estrictamente creciente,
derivable en el intervalo y que tenga una inversa, y comunmente a esta función
se le llama \textit{función de activación}.

Un posible ejemplo de la función de activación es la \textit{función logística},
que se define como $g(z) = \frac{1}{1 - e^{-z}}$. Para esta en particular, $g(z)' = g(z) [1 - g(z)]$, $g(-z) =   1 - g(z)$ y $g(z)^{-1} = \ln(\frac{z}{1 - z})$

\subsection{Fórmula de Bayes}

\todo[inline]{Explicar de dónde viene la fórmula de Bayes y poner
 el ejemplo de los Halcones y las Águilas }


\subsection{Clasificadores (Bayesianos) generativos}
Si tenemos dos clases ($K = 2$), la probabilidad de que el dato
$x$ pertenezca a la clase $C_1$ es:

\begin{eqnarray*}
 P(C_1 | x)
 &=&
 \frac{P(x | C_1) P(C_1)}
 {
  P(x | C_1)P(P_1) + P(x | C_2)P(C_2)
 } \\
 %
 &=&
 \frac{1}
 {
  1 + \frac
  {P(x | C_2)P(C_2)}
  {P(x | C_1)P(C_1)}
 }
\end{eqnarray*}

Si ahora definimos:

\begin{equation*}
 a(x) = \ln
 \frac
 {P(x | C_1)P(C_1)}
 {P(x | C_2)P(C_2)}
\end{equation*}

La probabilidad es:

\begin{equation*}
 P(C_1 | x) = \frac{1}
 {1 + exp(-a(x))}
\end{equation*}

Esta es una \textit{función logística} que se conoce como
``\textit{log of the odds}''

El caso genérico para $K \geq 2$ clases es:

\begin{eqnarray*}
 P(C_k | x)
 &=&
 \frac
 {P(x | C_k)P(C_k)}
 {
  \sum_{j = 1}^{K} P(x | C_j)P(C_j)
 } \\
 %
 &=&
 \frac
 {exp(a_k(x))}
 {\sum_{j = 1}^{K} P(x | C_j)P(C_j)}
\end{eqnarray*}

Ahora hemos generalizado la definición con:

\begin{equation*}
 a_k(x) = \ln
 P(x | C_k)P(C_k)
\end{equation*}


Esta función se llama \textit{softmax}, y es una generalización de la \textit{función logística}. Este tipo de modelos se llaman ``modelos generativos'', porque requieren conocer $P(x | C)$, y una vez sabes eso, eres capaz de generar nuevos datos en la clase $C$

\paragraph{Ejemplo: Generador Gaussiano}
Si asumimos que los datos de la clase $C_k$ siguen una distribución guassiana, entonces

\begin{equation*}
    P(x | C_k) =
    \frac{1}{(2\pi)^{\frac{d}{2}}}
    \frac{1}{|\Sigma_k|^{\frac{1}{2}}}
    exp\Big(
    -\frac{1}{2}
    (x - \mu_k)^T
    \Sigma_k^{-1}
    (x - \mu_k)
    \Big)
\end{equation*}

Para el caso $K = 2$ previamente hemos definido

\begin{equation*}
    a(x) = \ln
    \frac
    {
    P(x|C_1)P(C_1)
    }{P(x | C_2)P(C_2)}
\end{equation*}

Puesto que sabemos que los datos siguen una distribución normal, lo podemos escribir como:

\begin{align*}
    a(x) &= \ln(P(x|C_1)P(C_1)) - \ln(P(x | C_2)P(C_2)) \\
    %
    a(x) &=
    \begin{aligned}[t]
        &\ln(\cancel{\mcirc}) -\frac{1}{2}\ln(|\Sigma_1|) -\frac{1}{2}
        (x - \mu_1)^T
        \Sigma_1^{-1}
        (x - \mu_1) + \ln(P(C_1)) \\
        &-\ln(\cancel{\mcirc}) + \frac{1}{2}\ln(|\Sigma_2|) + \frac{1}{2}
        (x - \mu_2)^T
        \Sigma_2^{-1}
        (x - \mu_2) -\ln(P(C_2)) \\
    \end{aligned}\\
\end{align*}

Donde $\ln(\cancel{\mcirc}) = \ln \frac{1}{(2\pi)^{\frac{d}{2}}}$, que es constante en los dos términos y se anula.

Con esto ya podemos calcular $P(C_1|x)$. A este clasificador se le llama \textit{Quadratic Discriminant Analysis} (QDA).

Si se asume que $\Sigma_1 = \Sigma_2 = \Sigma$, entonces la ecuación anterior se simplifica, deja de ser cuadrática y es lineal:

\begin{equation*}
    a(x) = w^Tx + w_0
\end{equation*}

Donde $w = \Sigma^{-1} (\mu_1 - \mu_2)$ y $w_0 = \frac{1}{2}\mu_2^T\Sigma^{-1}\mu_2 + \frac{1}{2}\mu_1^T\Sigma^{-1}\mu_1 + \ln \frac{p(C_1)}{P(C_2)}$

Esta, como es lineal, se llama \textit{Linear Discriminant Analysis} (LDA)

¿Y qué ocurre cuando $K \geq 2$? Entonces se usa la definición

\begin{equation*}
    a_k(x) = \ln P(x|C_k)P(C_k)
\end{equation*}

Que sustituyendo queda como:

\begin{equation*}
    a_k(x) = -\frac{1}{2}\ln|\Sigma_k| - \frac{1}{2}(x - \mu_k)^T\Sigma_k^{-1}(x - \mu_k) + \ln P(C_k)
\end{equation*}

Y ésta es la fórmula general para QDA. Si se vuelve a asumir que todas las clases tienen la misma varianza, entonces $a_k(x) = w_k^Tx + w_0$, donde $w_k = \Sigma^{-1}\mu_k$ y $w_0 = \frac{1}{2}\mu_k^T\Sigma^{-1}\mu_k + \ln P(C_k)$. Ésta es la fórmula general para LDA.

\subsubsection{Cálculos en la práctica}

\todo[inline]{Explicar los problemas en QDA y LDA cuando hay más atributos que número de ejemplos o cuando las matrices de varianza no son invertibles.}
\todo[inline]{Explicar también RDA}


\subsection{Naive Bayes}
Cuando el problema de clasificación es multivariado, lo que queremos representar es:

\begin{equation*}
    P(x | C_k) = P(X_1 = x_1 \wedge X_2 = x_2 \wedge \dots \wedge X_d = X_d | C_k)
\end{equation*}

En probabilidades, se sabe que

\begin{equation*}
    P(X_1,X_2,X_3) = P(X_3 | X_1,X_2) P(X_2 | X_1) P(X_1)
\end{equation*}

Entonces se podría representar el problema como

\begin{equation*}
    P(x | C_k )P(C_k) =
    P(X_1 = x_1 | C_k)P(C_k)
    \prod_{j = 2}^{d} P(X_j = x_j | X_{j - 1} = x_{j - 1} \wedge \dots \wedge X_1 = x_1 \wedge C_k)
\end{equation*}
